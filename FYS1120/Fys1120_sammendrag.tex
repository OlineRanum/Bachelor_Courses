\documentclass[
12pt, reprint, aip, onecolumn, notitlepage
]{revtex4-1}
\usepackage{graphicx}% Include figure files
\usepackage{dcolumn}% Align table columns on decimal point
\usepackage[utf8]{inputenc}
\usepackage{enumitem}
\usepackage{xcolor}
\usepackage{physics}
\usepackage{caption}
\usepackage{listings}
\usepackage[most]{tcolorbox}
\colorlet{shadecolor}{pink!15}
\parindent 0.25in
\makeatother
\setlength\parindent{0pt}
\usepackage[left=3cm, right=3cm, top=1cm]{geometry}

\renewcommand{\baselinestretch}{1.3}
\setlength{\parskip}{10pt}





\usepackage{geometry}
\geometry{
	a4paper,	
	left=20mm,
	right = 20mm, 
	top=20mm,
}



\makeatletter
\newcommand*{\rom}[1]{\expandafter\@slowromancap\romannumeral #1@}
\makeatother


 
\begin{document}

	\title{Fys 1120\\ \normalsize Formler og sammendrag} 
	\maketitle 


\section{Pensum} 

\begin{table}[!h]
	\begin{tabular}{ l l }
		\textbf{Uke 1} & Coloumbs Lov, Superposisjon\\ 
	   \textbf{Uke 2} & Skalarpotensial, gradient, romladningstetthet
 	\end{tabular}
\end{table}

\section{Columbs Lov og elektrostatikk}





\section{Midtveiseksamen}
Liste over forventede formler: \\
Coloumbs lov

E-felt for punktladnings-, linjeladnings-, flateladnings- og romladnings-tetthet, Gauss lov, definisjon av skalarpotensial, og skalarpotensialet for punktladning, linje-, flate- og romladninstetthet, hvordan finne E fra skalarpotensialet, Gauss lov for dielektriske medier, grensebetingelser for D og E, Poissons og Laplace likning og numerisk løsning av denne, egenskaper til ideelle ledere, definisjon av kapasitans, strøm som integral over strømtetthet, ohms lov (J=sigma E), definisjonen av resistans, bevaringsloven for ladning på integral og differensial form, Kirchoffs strømlov, Biot-Savarts lov, Lorentz kraft på ladninger og strømelementer, Amperes lov.


\end{document}