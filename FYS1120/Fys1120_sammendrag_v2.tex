\documentclass[
12pt, reprint, aip, onecolumn, notitlepage
]{revtex4-1}
\usepackage{graphicx}% Include figure files
\usepackage{dcolumn}% Align table columns on decimal point
\usepackage[utf8]{inputenc}
\usepackage{enumitem}
\usepackage{xcolor}
\usepackage{physics}
\usepackage{caption}
\usepackage{listings}
\usepackage[most]{tcolorbox}
\colorlet{shadecolor}{pink!15}
\parindent 0.25in
\makeatother
\setlength\parindent{0pt}
\usepackage[left=3cm, right=3cm, top=1cm]{geometry}

\renewcommand{\baselinestretch}{1.3}
\setlength{\parskip}{10pt}





\usepackage{geometry}
\geometry{
	a4paper,	
	left=20mm,
	right = 20mm, 
	top=20mm,
}



\makeatletter
\newcommand*{\rom}[1]{\expandafter\@slowromancap\romannumeral #1@}
\makeatother


 
\begin{document}

	\title{Fys 1120\\ \normalsize Formler og sammendrag} 
	\maketitle 


\section{Pensum} 

\begin{table}[!h]
	\begin{tabular}{ l l }
		\textbf{Uke 1} & Coloumbs Lov, Superposisjon\\ 
	   \textbf{Uke 2} & Skalarpotensial, gradient, romladningstetthet \\
	   \textbf{Uke 3} & Gauss lov, Dielektriske medier, Gauss lov i et dielektrisk medium. Poissons likning. \\
	   \textbf{Uke 4} & Poisson's likning og lyn, ideelle ledere. kapasitans.\\
	   \textbf{Uke 5} & Energi i elektriske felt, strøm og strømtetthet\\
	   \textbf{Uke 6} & Ohms lov, resistans, effekttap, ladningsbevaring og Kirchoffs strømlov, magnetostatikk: Biot-Savarts lov, strømelement, Lorentz-kraft \\
	   \textbf{Uke 7} &  Hall effekt, magnetiske krefter og moment, magnetisk fluks, amperes lov.
 	\end{tabular}
\end{table}

\textbf{Til midtveis:} \\
Coloumbs lov, E-felt for punktladnings-, linjeladnings-, flateladnings- og romladnings-tetthet, Gauss lov, definisjon av skalarpotensial, og skalarpotensialet for punktladning, linje-, flate- og romladninstetthet, hvordan finne E fra skalarpotensialet, Gauss lov for dielektriske medier, grensebetingelser for D og E, Poissons og Laplace likning og numerisk løsning av denne, egenskaper til ideelle ledere, definisjon av kapasitans, strøm som integral over strømtetthet, ohms lov (J=sigma E), definisjonen av resistans, bevaringsloven for ladning på integral og differensial form, Kirchoffs strømlov, Biot-Savarts lov, Lorentz kraft på ladninger og strømelementer, Amperes lov.
\newpage 


\section{1.1 Elektrostatikk}
\subsection{Coulomb's lov}
The interaction between interchanging charged point particles:
\begin{equation}
	\mathbf{F}=\frac{q Q}{4 \pi \epsilon_{0}} \frac{\hat{R}}{R^{2}}
\end{equation}
where $\mathbf{R}=\mathbf{r}_{Q}-\mathbf{r}_{q}$ and $\hat{R}=\frac{\mathbf{R}}{R}$

\subsubsection{Charge}

Charges are fundamental, quantized properties of matter which can be positive, negative or zero. \\
Charges are mesuredin units of Coulomb, C. 
The charge of a proton is e = $10.602\times 10^{-19}$ C.
Charge is conserved. 
$\epsilon_0 = 8.85 \cdot 10^{-12} C^2N^{-1}m^{-2}$ is the permittivity in vacuum, and this version of Coulombs law is only valid in vaccum. 

\subsubsection{Point Charges}
\textit{Point charge} is a charged body, where the dimensions of the body is much smaller than typical distances between bodies of interest. 

\subsubsection{Sign of Columb's law}
The direction of the force depends on both charges, and is determined by their products. Equal signs yields repulisve force, and opposite yields attractive forcee.

\textbf{There is no self interaction for a charged point particle}

\subsection{Superposition principle}

The force on a point charge q at $r_q$ drom point charges $Q_1$ and $Q_2$ at $r_{Q1}$ and $r_{Q2}$ where $\mathbf{R}_{1}=\mathbf{r}_{Q 1}-\mathbf{r}_{q}$ and $\mathbf{R}_{2}=\mathbf{r}_{Q 2}-\mathbf{r}_{q}$

\begin{equation}
\mathbf{F}=\mathbf{F}_{1}+\mathbf{F}_{2}=\frac{q Q_{1}}{4 \pi \epsilon_{0}} \frac{\hat{R}_{1}}{R_{1}^{2}}+\frac{q Q_{2}}{4 \pi \epsilon_{0}} \frac{\hat{R}_{2}}{R_{2}^{2}}
\end{equation}

\section{1.2 The Electric Field}
If there is a set of charges $Q_i$, we say that they set up an electric field everywhere in space
\begin{equation}
\mathbf{E}=\frac{\mathbf{F}_{q}}{q} \quad(q \rightarrow 0) 
\end{equation}
\begin{equation}
\mathbf{F}=q \mathbf{E}
\end{equation}
\subsubsection{The Electric field from single point charge}
\begin{equation}
\frac{1}{4 \pi \epsilon_{0}} \frac{Q}{R^{2}} \hat{R}
\end{equation}

\subsubsection{Superposition principle for the electric field}
\begin{equation}
\mathbf{E}=\frac{1}{q} \sum_{i} \mathbf{F}_{i}=\sum_{i} \frac{\mathbf{F}_{i}}{q}=\sum_{i} \mathbf{E}_{i}=\sum_{i} \frac{1}{4 \pi \epsilon_{0}} \frac{Q_{i}}{R_{i}^{2}} \hat{R}_{i}
\end{equation}

\textbf{Dipol}: Består av to ladninger av samme størrelse, men forskjellig fortegn, plassert i nærheten av hverandre. Sett fra langt unna er nettoladningen 0, men det elektriske feltet er ikke 0.

\subsubsection{Continous distributions of charge: charge density}
Når det er svært mange punktladninger er det enklere å se på en ladningsfordeling, heller en mange individuelle punktladninger. 
\begin{equation}
\rho=\rho_{v}=\frac{d Q}{d v} \quad \text { volume charge density } [C/m^3]
\end{equation}
The charge in a volume v is then the sum of the charges inside the volume:
\begin{equation}
Q_{v}=\int_{v} \rho(\mathbf{r}) d v
\end{equation}
Videre har vi 
\begin{equation}
\rho_{a}=\sigma=\frac{d Q}{d A} \quad \text { surface charge density }
\end{equation}
\begin{equation}
\rho_{l}=\sigma=\frac{d Q}{d l} \quad \text { line charge density }
\end{equation}

\subsubsection{Elektrisk felt fra Volum ladning tetthet}
\begin{equation}
d \mathbf{E}=\frac{1}{4 \pi \epsilon_{0}} \frac{\rho\left(\mathbf{r}^{\prime}\right) d v}{R^{2}} \hat{R}
\end{equation}
Trenger å summere over alle mulige r verdier i volumet
\begin{equation}
\mathbf{E}=\int_{v} \frac{1}{4 \pi \epsilon_{0}} \frac{\rho\left(\mathbf{r}^{\prime}\right) d v^{\prime}}{R^{2}} \hat{R} = =\int_{v} \frac{1}{4 \pi \epsilon_{0}} \frac{\rho\left(\mathbf{r}^{\prime}\right) d v^{\prime}}{\left(\mathbf{r}^{\prime}-\mathbf{r}\right)^{2}} \frac{\mathbf{r}^{\prime}-\mathbf{r}}{\left|\mathbf{r}^{\prime}-\mathbf{r}\right|}
\end{equation}

Elektriske feltet fra en overflate tetthet
\begin{equation}
\mathbf{E}=\int_{A} \frac{1}{4 \pi \epsilon_{0}} \frac{\rho\left(\mathbf{r}^{\prime}\right) d A^{\prime}}{R^{2}} \hat{R}
\end{equation}
Elektriske feltet fra en linjeladnings tetthet:
\begin{equation}
\mathbf{E}=\int_{L} \frac{1}{4 \pi \epsilon_{0}} \frac{\rho\left(x^{\prime}\right) d x^{\prime}}{R^{2}} \hat{R}
\end{equation}
\subsubsection{Symmetribetraktninger}:
For eksempel: Gitt en ring ladning, så vil det være umiddelbart synlig at alle komponenter i x og y retning kanselerer, og at feltet kun vil ha en komponent i z-retning. Dette er kun gyldig på z-aksen----

\section{Examples}
\begin{itemize}
	\item Electric field from two charges (FN 1)
	\item Electric field from a dipole
	\item Visualizing a dipole field
	\item Elektric Field from a ring charge
\end{itemize}


\end{document}