\documentclass[
12pt, reprint, aip, onecolumn, notitlepage
]{revtex4-1}
\usepackage{graphicx}% Include figure files
\usepackage{dcolumn}% Align table columns on decimal point
\usepackage[utf8]{inputenc}
\usepackage{enumitem}
\usepackage{xcolor}
\usepackage{physics}
\usepackage{caption}
\usepackage{listings}
\usepackage[most]{tcolorbox}
\colorlet{shadecolor}{pink!15}
\parindent 0.25in
\makeatother
\setlength\parindent{0pt}
\usepackage[left=3cm, right=3cm, top=1cm]{geometry}

\renewcommand{\baselinestretch}{1.3}
\setlength{\parskip}{10pt}





\usepackage{geometry}
\geometry{
	a4paper,	
	left=20mm,
	right = 20mm, 
	top=20mm,
}



\makeatletter
\newcommand*{\rom}[1]{\expandafter\@slowromancap\romannumeral #1@}
\makeatother


 
\begin{document}

	\title{Fys 1120\\ \normalsize Formler og sammendrag} 
	\maketitle 


\section{Pensum} 

\begin{table}[!h]
	\begin{tabular}{ l l }
		\textbf{Uke 1} & Coloumbs Lov, Superposisjon\\ 
	   \textbf{Uke 2} & Skalarpotensial, gradient, romladningstetthet \\
	   \textbf{Uke 3} & Gauss lov, Dielektriske medier, Gauss lov i et dielektrisk medium. Poissons likning. \\
	   \textbf{Uke 4} & Poisson's likning og lyn, ideelle ledere. kapasitans.\\
	   \textbf{Uke 5} & Energi i elektriske felt, strøm og strømtetthet\\
	   \textbf{Uke 6} & Ohms lov, resistans, effekttap, ladningsbevaring og Kirchoffs strømlov, magnetostatikk: Biot-Savarts lov, strømelement, Lorentz-kraft \\
	   \textbf{Uke 7} &  Hall effekt, magnetiske krefter og moment, magnetisk fluks, amperes lov.
 	\end{tabular}
\end{table}

\textbf{Til midtveis:} \\
Coloumbs lov, E-felt for punktladnings-, linjeladnings-, flateladnings- og romladnings-tetthet, Gauss lov, definisjon av skalarpotensial, og skalarpotensialet for punktladning, linje-, flate- og romladninstetthet, hvordan finne E fra skalarpotensialet, Gauss lov for dielektriske medier, grensebetingelser for D og E, Poissons og Laplace likning og numerisk løsning av denne, egenskaper til ideelle ledere, definisjon av kapasitans, strøm som integral over strømtetthet, ohms lov (J=sigma E), definisjonen av resistans, bevaringsloven for ladning på integral og differensial form, Kirchoffs strømlov, Biot-Savarts lov, Lorentz kraft på ladninger og strømelementer, Amperes lov.
\newpage 


\section{1.1 Elektrostatikk}
\subsection{Coulomb's lov}
The interaction between interchanging charged point particles:
\begin{equation}
	\mathbf{F}=\frac{q Q}{4 \pi \epsilon_{0}} \frac{\hat{R}}{R^{2}}
\end{equation}
where $\mathbf{R}=\mathbf{r}_{Q}-\mathbf{r}_{q}$ and $\hat{R}=\frac{\mathbf{R}}{R}$

\subsubsection{Charge}

Charges are fundamental, quantized properties of matter which can be positive, negative or zero. \\
Charges are mesuredin units of Coulomb, C. 
The charge of a proton is e = $10.602\times 10^{-19}$ C.
Charge is conserved. 
$\epsilon_0 = 8.85 \cdot 10^{-12} C^2N^{-1}m^{-2}$ is the permittivity in vacuum, and this version of Coulombs law is only valid in vaccum. 

\subsubsection{Point Charges}
\textit{Point charge} is a charged body, where the dimensions of the body is much smaller than typical distances between bodies of interest. 

\subsubsection{Sign of Columb's law}
The direction of the force depends on both charges, and is determined by their products. Equal signs yields repulisve force, and opposite yields attractive forcee.

\textbf{There is no self interaction for a charged point particle}

\subsection{Superposition principle}

The force on a point charge q at $r_q$ drom point charges $Q_1$ and $Q_2$ at $r_{Q1}$ and $r_{Q2}$ where $\mathbf{R}_{1}=\mathbf{r}_{Q 1}-\mathbf{r}_{q}$ and $\mathbf{R}_{2}=\mathbf{r}_{Q 2}-\mathbf{r}_{q}$

\begin{equation}
\mathbf{F}=\mathbf{F}_{1}+\mathbf{F}_{2}=\frac{q Q_{1}}{4 \pi \epsilon_{0}} \frac{\hat{R}_{1}}{R_{1}^{2}}+\frac{q Q_{2}}{4 \pi \epsilon_{0}} \frac{\hat{R}_{2}}{R_{2}^{2}}
\end{equation}

\section{1.1.2 The Electric Field}
If there is a set of charges $Q_i$, we say that they set up an electric field everywhere in space
\begin{equation}
\mathbf{E}=\frac{\mathbf{F}_{q}}{q} \quad(q \rightarrow 0) 
\end{equation}
\begin{equation}
\mathbf{F}=q \mathbf{E}
\end{equation}
\subsubsection{The Electric field from single point charge}
\begin{equation}
\frac{1}{4 \pi \epsilon_{0}} \frac{Q}{R^{2}} \hat{R}
\end{equation}

\subsubsection{Superposition principle for the electric field}
\begin{equation}
\mathbf{E}=\frac{1}{q} \sum_{i} \mathbf{F}_{i}=\sum_{i} \frac{\mathbf{F}_{i}}{q}=\sum_{i} \mathbf{E}_{i}=\sum_{i} \frac{1}{4 \pi \epsilon_{0}} \frac{Q_{i}}{R_{i}^{2}} \hat{R}_{i}
\end{equation}

\textbf{Dipol}: Består av to ladninger av samme størrelse, men forskjellig fortegn, plassert i nærheten av hverandre. Sett fra langt unna er nettoladningen 0, men det elektriske feltet er ikke 0.

\subsubsection{Continous distributions of charge: charge density}
Når det er svært mange punktladninger er det enklere å se på en ladningsfordeling, heller en mange individuelle punktladninger. 
\begin{equation}
\rho=\rho_{v}=\frac{d Q}{d v} \quad \text { volume charge density } [C/m^3]
\end{equation}
The charge in a volume v is then the sum of the charges inside the volume:
\begin{equation}
Q_{v}=\int_{v} \rho(\mathbf{r}) d v
\end{equation}
Videre har vi 
\begin{equation}
\rho_{a}=\sigma=\frac{d Q}{d A} \quad \text { surface charge density }
\end{equation}
\begin{equation}
\rho_{l}=\sigma=\frac{d Q}{d l} \quad \text { line charge density }
\end{equation}

\subsubsection{Elektrisk felt fra Volum ladning tetthet}
\begin{equation}
d \mathbf{E}=\frac{1}{4 \pi \epsilon_{0}} \frac{\rho\left(\mathbf{r}^{\prime}\right) d v}{R^{2}} \hat{R}
\end{equation}
Trenger å summere over alle mulige r verdier i volumet
\begin{equation}
\mathbf{E}=\int_{v} \frac{1}{4 \pi \epsilon_{0}} \frac{\rho\left(\mathbf{r}^{\prime}\right) d v^{\prime}}{R^{2}} \hat{R} = =\int_{v} \frac{1}{4 \pi \epsilon_{0}} \frac{\rho\left(\mathbf{r}^{\prime}\right) d v^{\prime}}{\left(\mathbf{r}^{\prime}-\mathbf{r}\right)^{2}} \frac{\mathbf{r}^{\prime}-\mathbf{r}}{\left|\mathbf{r}^{\prime}-\mathbf{r}\right|}
\end{equation}

Elektriske feltet fra en overflate tetthet
\begin{equation}
\mathbf{E}=\int_{A} \frac{1}{4 \pi \epsilon_{0}} \frac{\rho\left(\mathbf{r}^{\prime}\right) d A^{\prime}}{R^{2}} \hat{R}
\end{equation}
Elektriske feltet fra en linjeladnings tetthet:
\begin{equation}
\mathbf{E}=\int_{L} \frac{1}{4 \pi \epsilon_{0}} \frac{\rho\left(x^{\prime}\right) d x^{\prime}}{R^{2}} \hat{R}
\end{equation}
\subsubsection{Symmetribetraktninger}:
For eksempel: Gitt en ring ladning, så vil det være umiddelbart synlig at alle komponenter i x og y retning kanselerer, og at feltet kun vil ha en komponent i z-retning. Dette er kun gyldig på z-aksen.

\section{1.2 Electric scalar potential}
The electric scalar potential is a sort of potential energy per unit charge. 
The work on charge q when the charge is moved along a path from A to B is
\begin{equation}
	W_{AB} = \int_{A}^{B}\mathbf{F}\cdot d\mathbf{r} 
\end{equation}
The path integral is independent of the path, due to the fact that electric forces are conservative. \\
Therefore, we can introduce a potential energy for a charge q in the force field set up by Q as the work needed to move the charge from a point A to a reference point.
\begin{equation}
	U_A = W_{A, ref} = U(r)
\end{equation}
This is valid for any charge distribution, and subjected to the superposition principle. We define the potential energy as 
\begin{equation}
U_A = \int_{A}^{ref}\mathbf{F}\cdot d\mathbf{l} =  q\int_{A}^{ref}\mathbf{E}\cdot d\mathbf{l} 
\end{equation}
We then introduce the scalar potential $V_A$ so that
\begin{equation}
	U_A = W_{AB} = qV_A
\end{equation}

The scalar potential in a point A at $\mathbf{r}$ is 
\begin{equation}
	V_A = V_{A,1} +V_{A,2} + ... = \sum_i V_{A,i}
\end{equation}
It is common to set the reference point in infinity, as long as the charge distribution has a finite extent.
Furthermore both the divergence and curl of E is zero in electro statistics.
\subsection{Kirchoff's law of voltages}
The sum of potential differences along a closed loop (a circuit) is zero
\begin{equation}
	V_{AB} + V_{BC} + V_{CA} = \oint\mathbf{E}\cdot d \mathbf{l} = 0
\end{equation} 
\subsection{Electric potential due to a given charge distribution}
A set of charges:
\begin{equation}
	V(\mathbf{r}) = \sum_i\dfrac{Q_i}{4\pi4\epsilon_0\abs{\mathbf{r}-\mathbf{r}_i}}
\end{equation}
A continuous distribution of charges
\begin{equation}
	V = \int_v \dfrac{pdv}{4\pi\epsilon_0R}
\end{equation}

\subsection{Relation between electric potential and electric field}
\begin{equation}
	V(\mathbf{r}) = \int_{r}^{ref}\mathbf{E} \cdot d\mathbf{l}
\end{equation}
\begin{equation}
	\mathbf{E} = -\grad{V}
\end{equation}
\subsection{Equipotential surfaces}
An equipotential surface is a surface where the potential is constant. The gradient to the potential is always normal to the equipotential surface, and points in the direction that V increases the fastest  Therefore, the electric field is normal to the equipotential surface but point in the direction that V decreases the fastest.

\section{1.3 Gauss' law}
Gauss' law is well suited for systems with a high degree of symmetry.\\
Gauss' law states that the flux of the electric field through a closed surface S is equal to the total charge in the volume inside the surface
\begin{equation}
	\oint_S \mathbf{E} \cdot d\mathbf{S} = \dfrac{Q_S}{\epsilon_0}
\end{equation}
S is any closed surface and Q is the net charge in the volume.
\begin{equation}
	Q_s = \sum_i Q_{i,in} \hspace{5mm} Q_s = \int_v p_v dv
\end{equation}
\subsection{Electric flux}
The electric flux through a small surface d$\mathbf{A}$ is defined as 
\begin{equation}
	d\Phi = \mathbf{E}\cdot d\mathbf{A} = EdA\cos\theta
\end{equation}
d$\mathbf{A}$ is the oriented surface element
\begin{equation}
	d\mathbf{A} = \hat{n}dA 
\end{equation}
Only the field normal to the surface contributes to the flux. The surface normal points outward of the volume.\\
Gauss law is more general than Columb's law, and is even valid for moving charges. 
\subsection{Applying Gauss' law}
A lot of applying Gauss' law is about finding a symmetry that makes the formula valid. First, one need to use symmetry arguments to simplify the description of the electric field. Second, on needs to find a surface S where the electric field was constant. 
\subsubsection{Gauss' law for a single charge}
Take a single charge in the origin. We expect that the field must be the same for all possible rotations of the system. We expect the system to have spherical symmetry. Then we realize that the field only can be directed in the radial direction, or else it would break the symmetry. In addition, we realize that the field can not have an angular dependency, because this would also break the symmetry. The field cannot be larger in one direction, because the orientation of the coordinate system is arbitrary Then we apply Gauss.
\subsubsection{Receipe for using Gauss law to find the electric field}
\begin{itemize}
	\item Find a set of surfaces that enclose a volume such that $E\cdot\hat{n}$ is constant
	on each such surface element. (It may be zero on some of the surfaces
	— zero is a constant!)
	\item This often requires that you find a simplfied description of the field
	in a chosen coordinate system, such as $E = E_r(r)\hat{r}$.
	\item Find the flux integral
	\item Use Gauss’ law to find the electric field as a function of charge and
	position.
	\item Notice that the surface does not have to enclose all the charges — it
	is allowed and indeed often necessary to chose a surface that contains
	only some of the charges. However, the electric field must be a constant
	on the surfaces you have chosen.
\end{itemize}
\subsection{Gauss' law on differential form}
\begin{equation}
	\grad \cdot \mathbf{E} = \dfrac{\rho}{\epsilon_0} 
\end{equation}
\section{1.4 Polarization and Dielectrics}
What happens with the electric field inside a material, the electric field will induce local charge displacements in an insulator, which induce an electric field. In conductors, the charges become mobile. This alters the total electric field. 
\subsection{Dielectrics}
Dielectric or an insulator is a material with very little free charges (like water, ceramic, plastics or glass). We discern between two types of dielectric: polar and non-polar.
\subsubsection*{Polar dielectric}
\textbf{Definition}: In a polar dielectric the electrons are distributed relative to the positive charges so that molecules behave as individual dipoles
\textbf{External E}: If the polar dielectric is subjected to an external electric field, the dipole molecules will tend to orient in the electric field with the positive part of the dipole
pointing in the direction of the local field. Without an applied electric
field, the dipoles will point in random directions, with no net eect, but
with an applied electric field, the dipoles will tend to align with the
field. The stronger the field, the stronger will be the alignment and the
stronger the net dipoles.\\
\textbf{Example}: water

\subsubsection*{Non-polar dielectric}
\textbf{Definition}: A non-polar dielectric has electrons distributed symmetrically around the positive charges. There is no net dipoles in the system when there is no applied electric field. However if we apply an external electric field, the electron clouds around atoms
or molecules tend to be displaced: The electrons will be displaced in a
direction opposite the electric field, whereas the nucleus is much less displaced. Each atom becomes a small dipole. 
\textbf{Example}:

\subsubsection{Polar and non-polar induce dipoles}
Formation of a net set of dipoles. Average dipole moment:
\begin{equation}
	P_{av} = \dfrac{\sum_i \mathbf{p}_{i in dv}}{N_{in dv}}
\end{equation}
The polarization vector $\mathbf{P}$ is the dipole moment per unit volume. 
The polarization vector depends on total field. \\
\subsubsection{Linearr dielectrics}
For a linear dielectric the polarization is proportional to the total electric field
\begin{equation}
	\mathbf{P} = \chi_e\epsilon_0\mathbf{E}
\end{equation}
$\chi_e$ is called the electric susceptibility, it has no unit.


Te the effect of an applied field is only a reorientation of dipoles
(for a polar dielectric), or a local displacement of the electron cloud
(non-polar dielectric), there is no change in the net charge in a volume.

polarization due to small dipoles with dipole moment 
\begin{equation}
	\mathbf{p} = Q\mathbf{d}
\end{equation}
\subsection{Gauss' law and total charge in a volume}
Includes both the free charges, and the bound charges that are the result of the total electric field. 
\begin{equation}
	\epsilon_0\oint_S \mathbf{E} \cdot d\mathbf{S} =Q_f + Q_b
\end{equation}
\begin{equation}
	\epsilon_0\oint_S \mathbf{E} \cdot d\mathbf{S} =Q_f - \oint_S \mathbf{P}\cdot d\mathbf{S}
\end{equation}
\begin{equation}
\oint_S (\epsilon_0\mathbf{E} +\mathbf{P})\cdot d\mathbf{S} =Q_f
\end{equation}
\subsubsection{Displacement field}
\begin{equation}
	\mathbf{D} = \epsilon_0\mathbf{E} +\mathbf{P}
\end{equation}
\subsubsection{Gauss law}
\begin{equation}
	\epsilon_0\oint_S \mathbf{D} \cdot d\mathbf{S} =Q_{\textnormal{free in S}}
\end{equation}
\subsubsection{Linearly polarized media}
\begin{equation}
	\mathbf{D} = \epsilon\mathbf{E}
\end{equation}
$\epsilon_r$ is the relative permittivity, and $\epsilon$ is the absolute permittivity. 
\subsection{Gauss' law on differential form}
\begin{equation}
	\grad \cdot \mathbf{D} = \rho_{free}
\end{equation}

Materials are dielectric only up to a given field strength. If E becomes to large, the material ionizes, and we get formation of free charges, and the material becomes a conductor. This limit is called the dielectric strength. 
\subsection{Surface charges}
\begin{equation}
	\rho_S = \mathbf{P}\cdot\hat{n}
\end{equation}

\subsection{Boundary conditions for E and D}
On the interface of two dielectric media, like plastic and water. For instance, we know the electric field E and displacement D on one side of the interface, what is it on the other? 
\subsubsection{Tangential boundary condition}
Use that the line integral of E over a closed loop is zero. 
\begin{equation}
	\mathbf{E_1}\cdot d\mathbf{l} + \mathbf{E_2}\cdot (-d\mathbf{l})= 0
\end{equation}
\begin{equation}
	E_{1t} = E_{2t}
\end{equation}

\subsubsection{Normal boundary condition}
Use Gauss' law to relate the normal component of the field on each side
\begin{equation}
	D_{1n}- D_{2n} = \sigma_s
\end{equation}
The normal components of D is related to thesurface charge density. 
\subsubsection{Two dielectric media}
If both media are dielectric, there are no free surface charges, and then 
\begin{equation}
	D_{1n} = D_{2n}
\end{equation}




\section{1.5 Laplace Equation}
Poissons and Laplace equation lets us find the electric potential and therefore the electric field by solving a differential equation. 

\subsection{Lightning and the effect of dielectric breakdown}
When the potential across a small dielectric region exceeds a maximum value, the charges (electrons) are noe long bound, but ripped from their bound position. We call this dielectric breakdown. The result is the motion of charges, and a change in potential. 
\subsection{Finding the electric potential}
Using the equations we already know
\begin{equation}
	\grad \cdot \mathbf{D} = \grad \cdot \epsilon \mathbf{E} = \grad \cdot (-\epsilon \grad{V})
\end{equation}
This gives Poisson's equation
\begin{equation}
	\grad^2V = -\dfrac{\rho}{\epsilon}
\end{equation}
Where Laplace equation
\begin{equation}
	\grad^2V = 0
\end{equation}

\subsection*{Existence, uniqueness and boundary conditions}
Poisson's equation has a unique solution given that the value of V on the external boundery is specified. If we find a solution that satisfies the boundary conditions, then the solution is the solution. 
\textbf{Dirichlet boundary conditions}: Specifies the calue of the potential at the boundary\\
\textbf{Von Neumann boundary conditions}: Specify the derivative of V in a given direction of n, typically normal to the boundary. 

\subsection*{Lightning}


\section{1.6 Conductors}

\textit{The motion of charges}\\

\textbf{Conductors}: Materials that contain a large number of charges, usually electrons, that are free to move. 

\subsection*{Properties of an ideal conductor in equilibrium}
\textbf{Equilibrium}: When all the charges have finished moving

\begin{enumerate}
	\item $\mathbf{E} = 0$ inside the ideal conductor. Otherwise, the electric field would move charges until they fall to rest. The field from the field from these charges will then compensate the external field.
	\item $\rho = 0$ inside an ideal conductory. Since $\mathbf{E} = 0$ inside the conductor, Gauss law tell us that $\grad{\cdot\mathbf{D}} = \grad{\cdot(\epsilon_0\mathbf{E} + \mathbf{P})} = 0$. All the charges must be on the surface. The surface charge density is often called an induced surface charge density, since it is often the consequence of an external field
	\item The conductor is an equipotential surface. (A potential s differences between two points in the conductor is $V_{AB} = \int_{A}^{B}\mathbf{E}\cdot d\mathbf{l} = \int_{A}^{B}0d\mathbf{l} = 0$
	\item $E_t = 0$ immediately outside the conductor This follows from the boundary conditions $E_{2t} = E_{1t}$, and $E_{t}$ is zero inside the conductor.
	\item $E_n = \sigma/\epsilon_0$ (Assume no polarization) from boundary conditions and that $E_n = 0$ inside the conductor. So $\mathbf{E}$ is normal to a conductor (that is linear dielectric, or not a dielectric). 
\end{enumerate}
OBS: There can be excess charge in a conductor, but it has to be on the surface. I.e. net charge does not have to be zero. 


\subsection{Speilladning}
A point charge Q is located a distance h above a grounded, conducting plane with potential $V_0 = 0$, how to find the surface charge distribution in the conductor plane? 



\section{1.7 Capacitance}
A capacitor is to conductors separated by a dielectric medium with permittivity $\epsilon$
\begin{equation}
	C = \dfrac{Q}{V}, \hspace{5mm} [F]
\end{equation}
Circuit equation for a capacitor
\begin{equation}
	I = \dfrac{dQ}{dt} = \dfrac{d(CV)}{dt} = C\dfrac{dV}{dt}
\end{equation}
\textbf{Current}: Charge per unit time
\begin{equation}
	I = \dfrac{dQ}{dt}
\end{equation}
\subsection{parallel-plate capacitor}
To parallel, equal, conducting planes of area S.
Given that the planes are separated by an isolator, a pure dielectric media, the capacitances is 
\subsubsection*{parallel coupling of parallel plate capacitor}
\begin{equation}
	C = \dfrac{Q_1 +Q_2 + \dots + Q_N}{V} = C_1 + C_2 +\dots C_N
\end{equation}
\subsubsection*{Series coupling of parallel plate capacitor}
\begin{equation}
	C = \dfrac{Q}{V} = \dfrac{Q}{V_{C1} + V_{C2} + \dots \ V_{CN}} = \dfrac{1}{\dfrac{1}{C_1}+\dfrac{1}{C_2}+\dots + \dfrac{1}{C_N}}
\end{equation}
Disse formlene gjelder kun dersom det ikke går feltlinjer mellom de ulike kondensatorene
\subsection{Capacitance per unit length of a coaxial cable}
One inner and one outer conductor
\begin{equation}
	C' = \dfrac{Q'}{V_0} \end{equation}
	\begin{equation}
	V_0 = \dfrac{Q'}{2\pi\epsilon}\ln{\dfrac{b}{a}}\end{equation}
\begin{equation}
	C' = \dfrac{2\pi\epsilon}{\ln{\dfrac{b}{a}}}
\end{equation}

\section{1.8 Energy in electrical fields}
How much energy is stored in a capacitor?
Assuming linear media around conductors. Initially we have zero potential difference or charge. Then vi move (positive) charge from the lower plane to the upper, until we have +Q charge and potential V on the upper plane. This process is initially easy, but becomes harder as we have to work against the electrical field. Therefore, we have to preform work.
\begin{equation}
	A_e = \int_{0}^{q}V(q)dq =\int_{0}^{q}\dfrac{q}{C}dq  = \dfrac{Q^2}{2C}
\end{equation}
This is the electrical energy stored in the capacitor, and is equivalent to 
\begin{equation}
	W_e = \dfrac{1}{2}CV^2
\end{equation}
Is only valid in a linear, isotropic media. The energy density:
\begin{equation}
	w_e = \dfrac{1}{2}\epsilon E^2 = \dfrac{1}{2}\mathbf{D}\cdot \mathbf{E}
\end{equation}
The latter is also valid in an anisotropic media. 

\section{1.9 Current, current density, loss, resistance and conservation of charge}
An electrical field exerts a force $q\mathbf{E}$ on a charge q, if the media has free charges a current will be induced. 
\begin{equation}
	\mathbf{J} = Nq\mathbf{v}
\end{equation}
with N various charge carriers
\begin{equation}
\mathbf{J} = \sum_i N_iq_i\mathbf{v}_i
\end{equation}
The current 
\begin{equation}
	I = \int_S \mathbf{J}\cdot d\mathbf{S}
\end{equation}
Ohm's law on local form
\begin{equation}
	\mathbf{J} = \sigma\mathbf{E}
\end{equation}
the conductivity $\sigma$ states how easily the conductor carries current. An ideal isolator has $\sigma = 0$ and an ideal conductor has $\sigma = \infty$
The efficiency loss per unit volume
\begin{equation}
	p_J = \mathbf{J}\cdot\mathbf{E}
\end{equation} 
Ohm's law on global form
\begin{equation}
	V = IR
\end{equation}
Efficiency loss in a resistance
\begin{equation}
	P_J = \int_v\mathbf{J}\cdot\mathbf{E}dv = V\dfrac{dQ}{dt}=VI = \dfrac{V^2}{R}
\end{equation} 
\subsection{Conservation of charge}
\begin{equation}
	I_s = -\dfrac{dQ}{dt} \equiv \oint_s \mathbf{J}\cdot \mathbf{S} = -\dfrac{d}{dt}\int_v\rho dv
\end{equation}
The minus sign indicates that we are looking at a current going out of the surface, and thus reducing Q. In the statics
\begin{equation}
	oint_s \mathbf{J}\cdot \mathbf{S} = 0
\end{equation}
This is known as Kirchoffs law of currents.
For a current hub with n conductors, this can also be stated as
\begin{equation}
	\sum_{i=1}^{n} I_i = 0
\end{equation}
Kirchoffs is only valid for constant currents. 
On differential form 
\begin{equation}
	\div{\mathbf{J}} = -\dfrac{\delta \rho}{\delta t}
\end{equation}

\section{2.0 Magnetostatics}
The forces between conducting conductors. We introduce the magnetic current density $\mathbf{B}$. 
\section{2.1 Magnetic force, Biot-Savarts law and currentelements}

\section{2.2 Magnetic forces and momentum}
\section{2.3 Magnetic flux out of a closed surface is zero}
\section{2.4} Amperes law for constant currents


















\end{document}